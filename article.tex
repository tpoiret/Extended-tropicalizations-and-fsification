\documentclass[a4paper,12pt]{amsart} %{article}
\title[label]{Title}

%GENERAL PACKAGES
%%%%%%%%%%%%%%%%%%%%%%%%%%%%%%%%%%%%%%%%%%%%%%%%
\usepackage{amsmath, amssymb, mathrsfs, amsthm, shorttoc, stmaryrd, tikz-cd, graphicx}%, txfonts}% amsfonts, url,, tikz, wrapfig, newclude, }
\usepackage{mathtools} % for \coloneqq (and presumably other stuff!)
\usepackage{comment}
\usepackage{hyperref}
\usepackage[all]{xy}
%\usepackage[left=2cm,right=2cm,top=3cm,bottom=3cm]{geometry}
\usepackage[all]{xypic}
%\renewcommand{\contentsname}{Table of contents}
%\renewcommand{\cftdotsep}{1}
%\renewcommand{\cftchapleader}{\bfseries\cftdotfill{\cftsecdotsep}}
%\usepackage{bbold}

\makeatletter
\providecommand{\leftsquigarrow}{%
  \mathrel{\mathpalette\reflect@squig\relax}%
}
\newcommand{\reflect@squig}[2]{%
  \reflectbox{$\m@th#1\rightsquigarrow$}%
}
\makeatother


%TOC
\setcounter{tocdepth}{2}% to get subsections in toc
 
\let\oldtocsection=\tocsection
 
\let\oldtocsubsection=\tocsubsection
 
\let\oldtocsubsubsection=\tocsubsubsection
 
\renewcommand{\tocsection}[2]{\hspace{0em}\oldtocsection{#1}{#2}}
\renewcommand{\tocsubsection}[2]{\hspace{1em}\oldtocsubsection{#1}{#2}}
\renewcommand{\tocsubsubsection}[2]{\hspace{2em}\oldtocsubsubsection{#1}{#2}}

%ENUMITEM
%%%%%%%%%%%%%%%%%%%%%%%%%%%%%%%%%%%%%%%%%%%%%%%%
\usepackage{enumitem}
% eg \begin{enumerate}[label={\ref{definition:quasisplit}.\arabic*}]
% eg. \begin{itemize}[label={1.\arabic*}]

%CLEVEREF STUFF	
%%%%%%%%%%%%%%%%%%%%%%%%%%%%%%%%%%%%%%%%%%%%%%%%
\usepackage{cleveref}
%\usepackage{nameref}
\crefformat{equation}{(#2#1#3)}
\let\oref\ref
\AtBeginDocument{\renewcommand{\ref}[1]{\cref{#1}}}
\numberwithin{equation}{subsection}

%TIKZ
%%%%%%%%%%%%%%%%%%%%%%%%%%%%%%%%%%%%%%%%%%%%%%%%
\usepackage{pgf,tikz}
\usetikzlibrary{matrix, calc, arrows}
%\usetikzlibrary{cd}
%\DeclareMathOperator{\coker}{coker}
\usepackage{tikz-cd} %this is the old version, but it also the one the arXiv supports at the moment (2016)
\usetikzlibrary{decorations.pathmorphing,shapes}
%TODONOTES
%%%%%%%%
%I found this clashed with several things, and it was easier just to do things by hand. 
%\usepackage{todonotes}
%\setlength{\marginparwidth}{3cm}
%\reversemarginpar

%Sites
%%%%%%%%%%%%%%%%%%%%%%%%%%%%%%%%%%%%%%%%%%%%%%%%
\newcommand{\Sch}[1]{\cat{Sch}/#1}
\newcommand{\LSch}{\cat{LSch}}
\newcommand{\LSchfs}{\cat{LSch}^{\on{fs}}}
\newcommand{\Schet}[1]{(\cat{Sch}/#1)_\et}
\newcommand{\LSchet}[1]{(\cat{LSch}/#1)_\et}
\newcommand{\shSchet}[1]{\on{Sh}(\cat{Sch}/#1)_\et}
\newcommand{\shLSchet}[1]{\on{Sh}(\cat{LSch}/#1)_\et}
\newcommand{\abSchet}[1]{\on{Ab}(\cat{Sch}/#1)_\et}
\newcommand{\abLSchet}[1]{\on{Ab}(\cat{LSch}/#1)_\et}
\newcommand{\sh}[1]{\on{Sh}(#1)}
\newcommand{\ab}[1]{\on{Ab}(#1)}
\renewcommand{\in}{\operatorname{in}}
\newcommand{\inleq}{\operatorname{in}_{\preceq}}


% DOUBLE AND TRIPLE ARROWS
%%%%%%%%%%%%%%%%%%%%%%%%%%%%%%%%%%%%%%%%%%%%%%%%
\makeatletter
\newcommand*{\doublerightarrow}[2]{\mathrel{
  \settowidth{\@tempdima}{$\scriptstyle#1$}
  \settowidth{\@tempdimb}{$\scriptstyle#2$}
  \ifdim\@tempdimb>\@tempdima \@tempdima=\@tempdimb\fi
  \mathop{\vcenter{
    \offinterlineskip\ialign{\hbox to\dimexpr\@tempdima+1em{##}\cr
    \rightarrowfill\cr\noalign{\kern.5ex}
    \rightarrowfill\cr}}}\limits^{\!#1}_{\!#2}}}
\newcommand*{\triplerightarrow}[1]{\mathrel{
  \settowidth{\@tempdima}{$\scriptstyle#1$}
  \mathop{\vcenter{
    \offinterlineskip\ialign{\hbox to\dimexpr\@tempdima+1em{##}\cr
    \rightarrowfill\cr\noalign{\kern.5ex}
    \rightarrowfill\cr\noalign{\kern.5ex}
    \rightarrowfill\cr}}}\limits^{\!#1}}}
\makeatother
%eg
%$A\doublerightarrow{a}{bcdefgh}B$
%$A\triplerightarrow{d_0,d_1,d_2}B$

%MATH OPERATORS
%%%%%%%%%%%%%%%%%%%%%%%%%%%%%%%%%%%%%%%%%%%%%%%%
\newcommand{\on}[1]{\operatorname{#1}}
\newcommand{\bb}[1]{{\mathbb{#1}}}
\newcommand{\cl}[1]{{\mathscr{#1}}}
\newcommand{\ca}[1]{{\mathcal{#1}}}
\newcommand{\bd}[1]{{\mathbf{#1}}}

\newcommand{\ul}[1]{{\underline{#1}}}

%SHEAF VERSIONS OF COMMON OPERATORS
%%%%%%%%%%%%%%%%%%%%%%%%%%%%%%%%%%%%%%%%%%%%%%%%
\def\sheaflie{\mathcal{L}ie}
\def\sheafhom{\mathcal{H}om}
\def\sheafisom{\mathcal{I}som}

\def\sheafaut{\mathcal{A}ut}
%BRACKETS AND PAIRINGS
%%%%%%%%%%%%%%%%%%%%%%%%%%%%%%%%%%%%%%%%%%%%%%%%
\def\lleft{\left<\!\left<}%{\langle\langle}
\def\rright{\right>\!\right>}%{\rangle\rangle}
\newcommand{\Span}[1]{\left<#1\right>}
\newcommand{\SSpan}[1]{\lleft#1\rright}
\newcommand{\abs}[1]{\lvert#1\rvert}
\newcommand{\aabs}[1]{\lvert\lvert#1\rvert\rvert}
\newcommand{\cat}[1]{\bd{#1}}

%SUB- AND SUPERSCRIPTS
%%%%%%%%%%%%%%%%%%%%%%%%%%%%%%%%%%%%%%%%%%%%%%%%
\newcommand{\algcl}{^{\textrm{alg}}}
\newcommand{\sepcl}{{^{\textrm{sep}}}}
\newcommand{\urcl}{{^{\textrm{ur}}}}
\newcommand{\ssm}{_{\smooth}}

%FONT ENCODING AND EXOTIC LETTERS
%%%%%%%%%%%%%%%%%%%%%%%%%%%%%%%%%%%%%%%%%%%%%%%%
%\usepackage[OT2,OT1]{fontenc} %These three lines to print Sha cleanly. 
%\DeclareSymbolFont{cyrletters}{OT2}{wncyr}{m}{n}
%\DeclareMathSymbol{\Sha}{\mathalpha}{cyrletters}{"58}

%MISC ARROWS
%%%%%%%%%%%%%%%%%%%%%%%%%%%%%%%%%%%%%%%%%%%%%%%%
\newcommand{\lra}{\longrightarrow}
\newcommand{\hra}{\hookrightarrow}
\newcommand{\sub}{\subseteq}
\newcommand{\tra}{\rightarrowtail}
\newcommand{\iso}{\stackrel{\sim}{\lra}}

%Sam's stuff
%%%%%%%%%%%%%%%%%%%%%%%%%%%%%%%%%%%%%%%%%%%%%%%%
\DeclareMathOperator*{\colim}{co{\lim}}
\DeclareMathOperator{\Ima}{Im}
\def\AF{{\mathcal{AF}}}
\def\sat{{\rm sat}}
\def\:{\colon}
\def\.{,\dots,}
\def\into{\hookrightarrow}
\newcommand{\oM}{{\overline{M}}}
\newcommand{\oN}{{\overline{N}}}
\newcommand{\oP}{{\overline{P}}}
\newcommand{\oQ}{{\overline{Q}}}
\newcommand{\oR}{{\overline{R}}}
\def\Aut{{\rm Aut}}
\def\toisom{\xrightarrow{{_\sim}}}
\def\CC{\mathbb C}
\def\ZZ{\mathbb Z}
\def\RR{\mathbb R}
\def\NN{\mathbb N}
\def\AA{\mathbb A}
\def\QQ{\mathbb Q}
\def\Zlog{\mathbb Z^{\log}}
\def\trop{\operatorname{trop}}
\def\coker{\operatorname{coker}}
\def\im{\operatorname{im}}
\def\et{\mathrm{\acute{e}t}}
\def\stet{\mathrm{st\acute{e}t}}
\def\overnorm#1{\overline{#1}}
\def\Et{\mathrm{\acute{E}t}}
\def\skel{\operatorname{skel}}
\def\Tropic{\operatorname{TroPic}}
\def\Logpic{\operatorname{LogPic}}
\def\sLPic{\operatorname{sLPic}}
\def\sTPic{\operatorname{sTPic}}
\def\sPic{\operatorname{sPic}}
\def\strLogPic{\operatorname{sLPic}}
\def\strTroPic{\operatorname{sTPic}}
\def\Pic{\operatorname{Pic}}
\def\Irr{\operatorname{Irr}}
\def\tor{\operatorname{tor}}
\def\Div{\operatorname{Div}}
\def\CDiv{\operatorname{CDiv}}
\def\BG{B\mathbb{G}_m}
\def\cL{\mathcal L}
\def\cO{\mathcal O}
\def\uX{{\underline{X}}}
\def\uS{{\underline{S}}}
\def\uY{{\underline{Y}}}
\def\L#1{\mathcal{L}og_{#1}}
\def\u#1{\underline{#1}}
\def\o#1{\overline{#1}}
\def\gp{\textrm{gp}}
\def\int{\textrm{int}}
\def\sat{\textrm{sat}}
\def\c#1{\mathcal{#1}}
\def\val#1{{#1}^{\rm{val}}}
\def\infroot#1{\sqrt[\infty]{#1}}
\def\inf#1{{#1}_{\infty}}
\def\infval#1{{{#1}^{\rm{val}}_{\infty}}}
\def\hello#1{$#1$}
\def\Perf#1{\textup{Perf}({#1})}
\newcommand{\cF}{{\mathcal F}}
\newcommand{\cZ}{{\mathcal Z}}
\newcommand{\cX}{{\mathcal X}}
\newcommand{\cU}{{\mathcal U}}
\newcommand{\Sh}{\operatorname{Sh}}
\newcommand{\PSh}{\operatorname{PSh}}
\newcommand{\HOM}{\operatorname{HOM}}
\newcommand{\Hom}{\operatorname{Hom}}
\newcommand{\Spec}{\operatorname{Spec}}
\newcommand{\Proj}{\operatorname{Proj}}
%\newcommand{\sslash}{\mathbin{/\mkern-6mu/}}
\newcommand{\invlim}{\displaystyle \lim_{ \longleftarrow } \,}
\newcommand{\dirlim}{\displaystyle \lim_{ \longrightarrow } \,}
\newcommand{\id}{id}
%THEOREMS
%%%%%%%%%%%%%%%%%%%%%%%%%%%%%%%%%%%%%%%%%%%%%%%%
\theoremstyle{definition}
\newtheorem{definition}{Definition}[section]
\newtheorem{condition}[definition]{Condition}
\newtheorem{problem}[definition]{Problem}
\newtheorem{situation}[definition]{Situation}
\newtheorem{construction}[definition]{Construction}
\newtheorem{property}[definition]{Property}
\newtheorem{fact}[definition]{Fact}

\theoremstyle{plain}% default
\newtheorem{conjecture}[definition]{Conjecture}
\newtheorem{question}[definition]{Question}
\newtheorem{proposition}[definition]{Proposition}
\newtheorem{lemma}[definition]{Lemma}
\newtheorem{theorem}[definition]{Theorem}
\newtheorem{corollary}[definition]{Corollary}
\newtheorem{claim}[definition]{Claim}
\newtheorem{inclaim}{Claim}[definition]


\theoremstyle{remark}

\newtheorem{remark}[definition]{Remark}
\newtheorem{aside}[definition]{Aside}
\newtheorem{example}[definition]{Example}


%MISC
%%%%%%%%%%%%%%%%%%%%%%%%%%%%%%%%%%%%%%%%%%%%%%%%
%\def\defeq{\stackrel{\text{\tiny def}}{=}}%depreciated: use coloneqq
\newcommand{\linequiv}{\sim_{\text{\tiny lin}}} %Linear equivalence
\renewcommand{\phi}{\varphi}
\newcommand{\cecH}{\check{\on H}}%For Cech cohomology


\newcommand{\Tcomment}[1]{{\color{red}T: #1}}

% \newcommand{\Dcomment}[1]{}
%temp
%%%%%%%%%%%%%%%%%%%%%%%%%%%%%%%%%%%%%%%%%%%%%%%%
\newcommand{\ZE}{\ca Z^\ca E}
\newcommand{\ZV}{\ca Z^\ca V}


%COMMENTS
%%%%%%%%%%%%%%%%%%%%%%%%%%%%%%%%%%%%%%%%%%%%%%%%
\newcounter{nootje}
\setcounter{nootje}{1}
\newcommand\todo[1]{[*\thenootje]\marginpar{\tiny\begin{minipage}
{20mm}\begin{flushleft}\thenootje : 
#1\end{flushleft}\end{minipage}}\addtocounter{nootje}{1}}
%\renewcommand{\todo}[1]{}

%SHORTCUTS FOR COMMON COMMANDS
%%%%%%%%%%%%%%%%%%%%%%%%%%%%%%%%%%%%%%%%%%%%%%%%
\newcommand{\mat}[1]{\begin{bmatrix}#1\end{bmatrix}}
\newcommand{\laurent}[2]{[#1_1^{\pm 1}, \dots, #1_{#2}^{\pm 1}]}

\begin{document}

\begin{abstract} 
Abstract
\end{abstract}

\maketitle
	
%\shorttableofcontents{Contents}{1}


\tableofcontents



\section{Introduction}

\section{Background and notation}
Tropical stuff, log stuff, Gr\"obner stuff. All monoids are commutative, all equivalence relations are monoidal (is this too confusing?), bars mean sharp, etc.

\section{The combinatorics of fsification}
\subsection{Monoids, binomial ideals and rewriting systems}

Throughout this section, unless specified otherwise, $\Phi\colon F \to M$ will be a surjective map of monoids with $F$ free of finite rank $n$, and $R\subset F\times F$ will be the equivalence relation defining $M$. We think of $F$ as the set of monomials (with coefficient $1$) in the coordinate ring of $\bb A^n$, and of $M$ as their restriction to the closed subvariety $Z\subset \bb A^n$ cut out by those binomials $f-f'$ with $(f,f')\in R$.

\begin{definition}
	A \emph{basis} for $M$ is a subset of $F\times F$ whose symmetric, transitive, monoidal closure is $R$. Let $\preceq$ be a monomial order on $F$. The \emph{initial term} $\inleq(a,b)$ of any $(a,b)\in F\times F$ is undefined if $a=b$, and $\max(a,b)$\footnote{The maximum is taken with respect to $\preceq$.} otherwise. A \emph{Gr\"obner basis} for $M$ (with respect to $\preceq$) is a basis $\{r_i=(a_i,b_i)\}_{1\leq i\leq k}$ such that $\in(R)=\bigcup\limits_i(\inleq(r_i)+F)$.
\end{definition}

For the remainder of this section, we fix a monomial order $\preceq$ on $F$. We will omit $\preceq$ from the notation when we can unambiguously do so.

\begin{lemma}[Macaulay]
	The restriction of $\Phi$ to $F\backslash\in(R)$ is bijective.
\end{lemma}

\begin{proof}
	If $a,b\in F$ have the same image under $\Phi|_{F\backslash\in(R)}$, then a fortiori $(a,b)$ is in $R$. Since neither $a$ nor $b$ may be in $\in(R)$, we have $a=b$. For surjectivity, pick any $m\in M$ and let $a$ be the minimum of $\Phi^{-1}(\{m\})$. By the minimality of $a$, $R$ contains no relation of the form $(a,b)$ with $b$ strictly smaller than $a$, i.e. $a$ is not in $\in(R)$.
\end{proof}

\begin{comment}
	Too detailed?
\begin{definition}
	Let $M$ be a monoid and $R$ a subset of $M\times M$. We say $R$ is \emph{$M$-stable} if for all $m\in M$ we have $(m,m)+R\subset R$. The \emph{monoidal equivalence relation generated by $R$} is the smallest $M$-stable equivalence relation $\tilde R\subset M\times M$ containing $R$. The \emph{monoidal quotient of $M$ by $R$} is the monoid $M/\tilde R$, which comes with a surjective monoid map $M \to M/\tilde R$. We will almost always simply call it the \emph{quotient of $M$ by $R$} and denote it by $M/R$.
\end{definition}

\begin{remark}
	If $M \to N$ is a surjective monoid map, the corresponding equivalence relation on $M$ is $M$-stable, so there is a natural contravariant equivalence between surjective morphisms $M \to N$ and $M$-stable equivalence relations on $M\times M$ ordered by inclusion.
\end{remark}
\end{comment}


\begin{lemma}
	A finitely generated monoid $M$ is finitely presented.
\end{lemma}

\begin{proof}
	Done in Redei's book, but maybe we need more. Given $F\twoheadrightarrow M$ and a monomial order on $F$, we will need existence and uniqueness of a reduced Gr\"obner basis for $M$.
\end{proof}


\begin{definition}
	Let $M$ be a monoid. The category of monoid maps $M \to N$ where $N$ is integral (resp. integral and saturated, resp. a group) admits an initial object $M \to M^\int$ (resp. $M \to M^\sat$, resp. $M \to M^\gp$). Its target monoid is called the \emph{integralification} (resp. \emph{integral saturation}, resp. \emph{groupification}) of $M$.
\end{definition}

\begin{lemma}
	Let $F \to M$ be a surjective monoid map with $F$ free and finitely generated. Pick a monomial order on $F$, and let $\{r_i=(a_i,b_i)\}_{1\leq i\leq n}$ be the reduced Gr\"obner basis defining $M$ in $F$. For each $i$, put $r'_i=(a'_i,b'_i)$, where $a'_i=a_i-(a_i\wedge b_i)$ and $b'_i=b_i-(a_i\wedge b_i)$. Then the $r_i'$ form a reduced Gr\"obner basis for $M^\int$.
\end{lemma}

\begin{proof}
	Let $R,R^\int$ be the relations defining $M$ and $M^\int$ in $F$. The $r'_i$ are in $R^\int$. Pick any relation $(a,b)$ in $R^\int$. We will show $(a,b)$ is generated by the $r'_i$. Without loss of generality, we may assume $a\wedge b=0$. There exists some $c\in F$ such that $(a+c,b+c)$ is in $R$. We may write a finite chain of substitutions \ref{to be made precise} $a+c=:\lambda^0 \to \dots \to \lambda^k:=b+c$ using the $r_i$. For any $0\leq j<k$, there is some $i$ such that $\lambda^j \to \lambda^{j+1}$ via $r_i$ i.e. $\lambda^j=\mu+a_i$ and $\lambda^{j+1}=\mu+b_i$.
\end{proof}


\subsection{Fsifying monoids}

\subsection{Fsifying log schemes}
The inclusion of categories $\LSchfs \to \LSch$ has a \ref{left or right or whatever} adjoint, called \emph{fsification}. When $X$ is a log scheme, we denote by $X^{fs}$ the fsification of $X$.

\section{Toric fibre products}

\section{Decomposing logarithmic fibre products}

\section{Examples}


\bibliographystyle{alpha} %amsplain}
\bibliography{biblio}


\bigskip

\noindent
Thibault POIRET,
{\sc University of Cambridge, United Kingdom} \\
Email address: {\tt tp528@ac.cam.uk}

\end{document}
